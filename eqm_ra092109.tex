\documentclass[12pt,twoside,a4paper]{article}
\usepackage[brazil,english]{babel}
\usepackage[utf8]{inputenc}
\usepackage[T1]{fontenc}
\usepackage{timbre-ic}
\usepackage{booktabs}
\usepackage[table]{xcolor}
\usepackage{url}

\begin{document}

\vskip 15mm

\begin{center} 
\textbf{Treplica wait-free implementation}
\end{center}

\vskip 5mm

\textbf{Orientando:} Luísa Madeira Cardoso

\textbf{Orientador:} Luiz Eduardo Buzato 

\vskip 20mm

\begin{abstract}

\end{abstract}

% resetando configs de layout
\newpage
\pagestyle{plain}
\headheight 0.0cm
\headsep 0.0cm
\footskip 2.2cm

\section{Introduction}


\section{Related Work}
\label{sec:related}

Wait-free data structures are the foundations of this work. They designate objects that are shared by concurrent processes and guarantee that each process will complete an operation in a finite number of steps. The later condition assures that, despite of failures, all processes will make progress. In other words, a starvation scenario is impossible. 

These structures have been widely studied in the last decades, since they are the heart of many important problems in the concurrent programming area \cite{herlihy2011art}.
The name "wait-free" may give the impression that such implementations are faster than its blocking counterparts. However, this notion is misleading.  Such algorithms can be complex and inefficient \cite{attiya1994wait}.  

In 1991, Maurice Herlihy showed a fundamental result that provided a technique to prove if a given wait-free implementation was possible or not \cite{herlihy1991wait}. This work also proved the existence of universal primitives, such as \textit{Compare and Swap}, that could be used to construct any sequential wait-free structure. 

Java provides a wait-free queue called \textit{ConcurrentLinkedQueue} that is based on a work by Michael and Scott \cite{michael1996simple}. This implementation was very promise because it showed the best performance in comparison to other blocking and non-blocking counterparts in the preliminary results presented by the authors. 


\section{Expected Contribution}
\label{sec:contrib}


 
\section{Schedule}
\label{sec:schedule}


\vskip 15mm

\bibliography{refs}{}
\bibliographystyle{acm}

\end{document}
